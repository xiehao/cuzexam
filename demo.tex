\documentclass[]{cuzexam}
% \documentclass[answer]{cuzexam}
% \documentclass[sheet]{cuzexam}
% \documentclass[answer,sheet]{cuzexam}

% =================================================
%       PDF信息
% =================================================
% PDF信息里的标题栏
\title{xxxx学期xxx期末考试x卷}
% PDF信息里的作者栏
\author{无名}
% PDF信息里的主题
\cuzsubject{无题}
% PDF信息里的关键词
\cuzkeywords{xxx; 期末}

% =================================================
%       试卷头信息
% =================================================
% 学年信息
\cuzacademicyear{20ab—20cd}
% 学期信息
\cuzsemester{z}
% 课程名称
\cuzcourse{\LaTeX{}入门}
% 任课教师
\cuzlecturers{无名甲、无名乙、无名}
% 考试类型,如:考试/考察
\cuzexamtype{考试}
% 试卷类型,如:A/B
\cuzpapertype{X}
% 计分表中大题的数目
\cuznumberparts{8}

\begin{document}

% 生成试卷表头
\makehead

\begin{extracomment}
    \textbf{可选语句:“注意:请将答案写在答题纸上,在试卷内作答无效。”}
\end{extracomment}

\makepart{判断题}{对的划\yes,错的划\no;共~5~小题,每小题~2~分,共~10~分}

\begin{problem}
    这是一个正确的结论。\pickout{\yes}
\end{problem}

\begin{problem}
    这是一个错误的结论。\pickout{\no}
\end{problem}

\begin{problem}
    这是一个正确的结论。\pickout{\yes}
\end{problem}

\begin{problem}
    这是一个错误的结论。\pickout{\no}
\end{problem}

\begin{problem}
    这是一个正确的结论。\pickout{\yes}
\end{problem}

% 用于在答题纸上生成对应区域,格式随意,此处画成表格
\begin{sheetcontents}
    \begin{tabularx}{\textwidth}{|*{5}{X<{\centering}|}}
        \hline
        \repeatcell{5}{text=#1} \\
        \hline
        \pickoutx{\yes} & \pickoutx{\no} & \pickoutx{\yes} & \pickoutx{\no} & \pickoutx{\yes} \\
        \hline
    \end{tabularx}
\end{sheetcontents}

\makepart{单选题}{共~5~小题,每小题~2~分,共~10~分}

\begin{problem}
    这是一个单选题,选啥看着办\pickout{A}
    \options{短选项甲}
            {短选项乙}
            {短选项丙}
            {短选项丁}
\end{problem}

\begin{problem}
    这是一个单选题,选啥看着办\pickout{B}
    \options{中等长度的选项甲}
            {中等长度的选项乙}
            {中等长度的选项丙}
            {中等长度的选项丁}
\end{problem}

\begin{problem}
    这是一个单选题,选啥看着办\pickout{C}
    \options{一个特别特别长的需要分四行显示的选项甲}
            {一个特别特别长的需要分四行显示的选项乙}
            {一个特别特别长的需要分四行显示的选项丙}
            {一个特别特别长的需要分四行显示的选项丁}
\end{problem}

\begin{problem}
    这是一个单选题,选啥看着办\pickout{A}
    \options{短选项甲}
            {短选项乙}
            {短选项丙}
            {短选项丁}
\end{problem}

\begin{problem}
    这是一个单选题,选啥看着办\pickout{B}
    \options{中等长度的选项甲}
            {中等长度的选项乙}
            {中等长度的选项丙}
            {中等长度的选项丁}
\end{problem}

\begin{problem}
    这是一个单选题,选啥看着办\pickout{C}
    \options{一个特别特别长的需要分四行显示的选项甲}
            {一个特别特别长的需要分四行显示的选项乙}
            {一个特别特别长的需要分四行显示的选项丙}
            {一个特别特别长的需要分四行显示的选项丁}
\end{problem}

\begin{problem}
    这是一个单选题,选啥看着办\pickout{A}
    \options{短选项甲}
            {短选项乙}
            {短选项丙}
            {短选项丁}
\end{problem}

\begin{problem}
    这是一个单选题,选啥看着办\pickout{B}
    \options{中等长度的选项甲}
            {中等长度的选项乙}
            {中等长度的选项丙}
            {中等长度的选项丁}
\end{problem}

\begin{problem}
    这是一个单选题,选啥看着办\pickout{C}
    \options{一个特别特别长的需要分四行显示的选项甲}
            {一个特别特别长的需要分四行显示的选项乙}
            {一个特别特别长的需要分四行显示的选项丙}
            {一个特别特别长的需要分四行显示的选项丁}
\end{problem}

\begin{problem}
    这是一个单选题,选啥看着办\pickout{A}
    \options{短选项甲}
            {短选项乙}
            {短选项丙}
            {短选项丁}
\end{problem}

\begin{sheetcontents}
    \begin{tabularx}{\textwidth}{|*{10}{X<{\centering}|}}
        \hline
        \repeatcell{5}{text=#1} \\
        \hline
        \pickoutx{A} & \pickoutx{B} & \pickoutx{C} & \pickoutx{D} & \pickoutx{A} \\
        \hline
        \repeatcell{5}{start=\lastvalue,text=#1} \\
        \hline
        \pickoutx{A} & \pickoutx{B} & \pickoutx{C} & \pickoutx{D} & \pickoutx{A} \\
        \hline
    \end{tabularx}
\end{sheetcontents}

\makepart{填空题}{共~5~空,每空~2~分,共~10~分}

\begin{problem}
    这是一道填空题,空在这:\fillindex{待显示的答案}。
\end{problem}

\begin{problem}
    这又是一道填空题,一道题可以有任意多个空,如\fillindex{又一个待显示的答案}与\fillindex{第三个待显示的答案},序号会自动增加。
\end{problem}

\begin{problem}
    最后一道填空题了,来个程序填空吧:
    \begin{minted}[escapeinside=@@]{cpp}
        template <@\fillindex{\texttt{typename T}}@>
        T add_two(T a, T b) {
            return @\fillindex{\texttt{a + b}}@;
        }
    \end{minted}
\end{problem}

\begin{sheetcontents}
    \begin{enumerate}[label=(\arabic*)]
        \item \fillinx{0.45}{待显示的答案}
        \item \fillinx{0.45}{又一个待显示的答案}
        \item \fillinx{0.45}{第三个待显示的答案}
        \item \fillinx{0.45}{\mintinline{cpp}{typename T}}
        \item \fillinx{0.45}{\mintinline{cpp}{a + b}}
    \end{enumerate}
\end{sheetcontents}

\makepart{简答题}{共~5~小题,每小题~4~分,共~20~分}

\begin{problem}
    第一道题的描述。
\end{problem}

\begin{solution}
    \begin{enumerate}[label=(\arabic*)]
        \item 第一条,随便说点什么;\dotfill 2分
        \item 第二条,再随便说点什么。\dotfill 2分
    \end{enumerate}
\end{solution}

\begin{studentanswer}
    \vspace{28ex}
\end{studentanswer}

\begin{problem}
    第二道题的描述。
\end{problem}

\begin{solution}
    \begin{enumerate}[label=(\arabic*)]
        \item 第一条,随便说点什么;\dotfill 2分
        \item 第二条,再随便说点什么。\dotfill 2分
    \end{enumerate}
\end{solution}

\begin{studentanswer}
    \vspace{28ex}
\end{studentanswer}

\begin{problem}
    第五道题的描述。
\end{problem}

\begin{solution}
    \begin{enumerate}[label=(\arabic*)]
        \item 第一条,随便说点什么;\dotfill 2分
        \item 第二条,再随便说点什么。\dotfill 2分
    \end{enumerate}
\end{solution}

\begin{studentanswer}
    \vspace{28ex}
\end{studentanswer}

\begin{problem}
    第三道题的描述。
\end{problem}

\begin{solution}
    \begin{enumerate}[label=(\arabic*)]
        \item 第一条,随便说点什么;\dotfill 2分
        \item 第二条,再随便说点什么。\dotfill 2分
    \end{enumerate}
\end{solution}

\begin{studentanswer}
    \vspace{28ex}
\end{studentanswer}

\begin{problem}
    第四道题的描述。
\end{problem}

\begin{solution}
    \begin{enumerate}[label=(\arabic*)]
        \item 第一条,随便说点什么;\dotfill 2分
        \item 第二条,再随便说点什么。\dotfill 2分
    \end{enumerate}
\end{solution}

\begin{studentanswer}
    \vspace{28ex}
\end{studentanswer}

\makepart{程序阅读题}{共~2~空,每空~5~分,共~10~分}

\begin{problem}
    写出下列程序的运行结果,按照输出结果依次填写~1~行输出。
    \begin{minted}{cpp}
        #include <iostream>
        
        int main() {
            std::cout << "Hello" << std::endl;
            return 0;
        }
        
    \end{minted}
    运行结果:
    \begin{enumerate}[label=(\arabic*),series=cuzafter]
        \item \fillinx{0.9}{\mintinline{console}{Hello}}
    \end{enumerate}
\end{problem}

\begin{problem}
    写出下列程序的运行结果,按照输出结果依次填写~1~行输出。
    \begin{minted}{cpp}
        #include <iostream>
        
        int main() {
            std::cout << "World" << std::endl;
            return 0;
        }
    \end{minted}
    运行结果:
    \begin{enumerate}[label=(\arabic*),resume=cuzafter]
        \item \fillinx{0.9}{\mintinline{console}{Hello}}
    \end{enumerate}
\end{problem}

\begin{sheetcontents}
    \begin{enumerate}[label=(\arabic*)]
        \item \fillinx{0.9}{\mintinline{console}{Hello}}
    \end{enumerate}
\end{sheetcontents}

\makepart{程序设计题}{共~2~题,第~1~题~8~分,第~2~题~12~分,共~20~分}

\begin{problem}
    编写哈喽世界。
\end{problem}

\begin{solution}
    \begin{minted}[escapeinside=@@,texcomments=true]{cpp}
        #include <iostream> \dotfill{5分}
        
        int main() {
            std::cout << "Hello World" << std::endl; @\dotfill{5分}@
            return 0;
        }
    \end{minted}
\end{solution}

\begin{studentanswer}
    \newpage
\end{studentanswer}

\begin{problem}
    编写古德白世界。
\end{problem}

\begin{solution}
    \begin{minted}[escapeinside=@@,texcomments=true]{cpp}
        #include <iostream> \dotfill{5分}
        
        int main() {
            std::cout << "Goodbye World" << std::endl; @\dotfill{5分}@
            return 0;
        }
    \end{minted}
\end{solution}

\begin{studentanswer}
    \newpage
\end{studentanswer}

\makepart{计算题}{共~1~题,每题~10~分,共~10~分}

\begin{problem}
    求解不定积分:
    $$f(x)=\int\sin{x}\mathrm{d}x$$
\end{problem}

\begin{solution}
    $$f(x)=\int\sin{x}\mathrm{d}x=\cos{x}+C$$
\end{solution}

\begin{studentanswer}
    \newpage
\end{studentanswer}


\makepart{证明题}{共~1~题,每题~10~分,共~10~分}

\begin{problem}
    证明:
    $$f(x)=\int\sin{x}\mathrm{d}x$$
\end{problem}

\begin{proof}
    $$f(x)=\int\sin{x}\mathrm{d}x=\cos{x}+C$$
\end{proof}

\begin{studentanswer}
    \newpage
\end{studentanswer}

\end{document}
